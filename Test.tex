%\documentclass[a4paper,10pt]{article}
%\documentclass[a4paper,10pt]{scrartcl}
\documentclass[a4paper,12pt,onecolumn]{article}

\usepackage[utf8]{inputenc}
\usepackage[T2A,T1]{fontenc}
\usepackage[russian]{babel} 
\usepackage{setspace}
\usepackage{amsmath}
\usepackage{cite}
\usepackage{amssymb}
% \usepackage{psfrag}
% \usepackage{pstricks}
\usepackage{epsfig}
\usepackage{graphicx}
\usepackage{subfig}
\usepackage[top=2cm, bottom=2cm, left=2cm, right=2cm]{geometry}

\newcommand{\eq}[1]      {(\ref{#1})}
\newcommand{\p}{\partial}
\newcommand{\rd}{\,\mathrm{d}}

\begin{document}
\title {Тест проверки остаточных знаний для магистров
}
\date{}
\maketitle
\noindent 1. Найти период функции
$$
y=3\sin (4\pi x+5).
$$
Для этой функции период можно найти по формуле:
\[T = \frac{2\pi}{|4\pi|}.\]

Упрощаем выражение:
\[T = \frac{2\pi}{4\pi} = \frac{1}{2}.\]

Таким образом, период этой функции равен $\frac{1}{2}$.

2. Найти общее решение дифференциального уравнения
$$
y''+4y'+8y=0.
$$
3. Вычислить интеграл
$$
\int\cfrac{2\rd x}{3x+5}.
$$
4. Найти производную
$$
y=\sqrt{\tg x}.
$$
так как это сложная функция, то мы умножаем производную внешней функции на производную внутренней и получаем в данном случае:
единица деленая на два корня тангенс x, умноженная на единицу, деленную на косинус квадрат икс
5. Найти интервалы возрастания и точки экстремума функции
$$
y=2x^3-6x^2-18x+22
$$
для нахождения экстремума и интервалов возрастания функции нужно найти точки, в которых производная будет ровна 0. найдём производную и приравняем её к нулю:
$$
y'=6x^2-12x-18=0
$$
найдём дескреминант получившегося квадратного уравнения по формуле
$$
b^2-4(ac)
$$
где a, b, c- коеффициенты квадратного уравнения. тогда дескреминант получится $$ d = 36 $$
найдём корни квадратного уравнения. по формуле $$ \cfrac{sqrt{\d} +-(b)}{2a}
получаем: $$ x1 = 1, x2 = 3 $$, это и есть экстремумы функции.
зная экстремумы можно сказать когда функция ростёт, а когда убывает данная функция ростёт от минус бесконечности до 1 и от 3 до + бесконечности.
6. Решить систему уравнений наибольшим числом способов
$$
  \left\{
    \begin{aligned}
      2x+3y=12,\\
      3x+2y=13.            
    \end{aligned}
  \right. 	
$$
данную систему уравнений можно решить методом краммера. составим матрицу для этого метода:
2 3 12
3 2 13
12 и 13- свободные члены уравнения. дальше нужно найти определитель матрицы, составленной только из коеффициентов, потом заменить 1-ый столбец данной матрицы на свободные члены, найти определитель получившейся матрицы и поделить найденный определитель на определитель матрицы коеффициентов, найдя таким образом первое неизвестное. повторив это с матрицей с заменённым вторым столбцом найдём второе неизвестное. в данном случае x = 3, y = 2

7. Исследовать  функцию на непрерывность
$$
y=\cfrac{x-1}{2+2^{1/x}}+\cfrac{|x+2|}{x^2+5x+6}.
$$

8. Найти предел
$$
\lim_{n\to\infty} \cfrac{3n +4}{n^2+n+11}
$$
так как числитель здесь ростёт быстрее знаменателя, придел будет равен нулю.
9. На ГенАссамблее ООН присутствут 142 страны, делегаты которых говорят на разных языках. Сколько переводчиков нужно, чтобы обеспечит перевод между любыми  делегациями? (Переводчик для перевода, например, между русскоязычной и англоязычной делегацией один. Он обеспечивает перевод как с английского, так и с русского языка). 
пусть n = количество делегаций, тогда количество переводчиков можно найти по следующей формуле:
$$
\cfrac{n*(n -1)}{2}
$$
тогда число переводчиков будет 1011
10. Вокруг  круглого стола расставлены 6 стульев. Какова вероятность, что из  6 человек, которые случайным образом рассаживаются по этим стульям, двое избранных сядут рядом? 
для решения этой задачи мы можем посадить одного человека на стул и рассаживать Вокруг него остальных. так как у нас останется 5 стульев и 2 из них находятся рядом с нужным человеком, то шанс того, что второй человек сядет на стул рядом с первым будут 0,4.
\end{document}